\documentclass{article}

\usepackage{ctex}
\usepackage{amsmath}

\begin{document}
\title{使用函数}
\author{\kaishu{贾泽昊}}
\date{\today}
\maketitle

\section{函数}
使用python求解整数规划,使用pulp库的函数

\subsection{lpproblem}
lpproblem函数用于创建一个线性规划问题,返回一个线性规划问题的实例,
该实例包含一个线性规划问题,一个线性约束集合和一个变量集合.
\begin{center}
\large Lpproblem(name,sense)
\end{center}
\par name:线性规划问题的名称
\par sense:线性规划问题的目标函数,可以是min或者max

\subsection{LpVariable}
LpVariable函数用于创建一个线性规划问题的变量,返回一个线性规划问题的变量,
\begin{center}
\large LpVariable(name,lowBound,upBound,cat)
\end{center}
\par name:变量的名称
\par lowBound:变量的下界
\par upBound:变量的上界
\par cat:变量的类型,可以为continuous(连续类型), integer(整数类型), binary(01类型)

\end{document}