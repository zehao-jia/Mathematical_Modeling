\documentclass{article}

\usepackage{ctex}
\usepackage{amsmath}
\usepackage{listings}
\usepackage{xcolor}

% 设置 Python 代码高亮
\lstdefinestyle{pythonstyle}{
    language=Python,
    basicstyle=\ttfamily,
    keywordstyle=\color{blue},
    commentstyle=\color{green!50!black},
    stringstyle=\color{red},
    numbers=left,
    numberstyle=\tiny\color{gray},
    stepnumber=1,
    numbersep=5pt,
    backgroundcolor=\color{gray!10},
    showspaces=false,
    showstringspaces=false,
    showtabs=false,
    tabsize=2
}

\begin{document}

\section{Scipy 中的非线性规划:minimize 函数}

在科学计算和优化问题中,\texttt{scipy} 库提供了强大的工具来解决非线性规划问题。其中,\texttt{scipy.optimize.minimize} 函数是一个非常实用的工具,它可以用于求解带约束和不带约束的非线性优化问题。

\subsection{函数原型}
\texttt{scipy.optimize.minimize} 函数的原型如下:
\begin{lstlisting}[style=pythonstyle]
scipy.optimize.minimize(fun, x0, args=(), method=None, jac=None, hess=None, hessp=None, bounds=None, constraints=(), tol=None, callback=None, options=None)
\end{lstlisting}
各参数的含义如下:
\begin{itemize}
    \item \texttt{fun}:需要最小化的目标函数。
    \item \texttt{x0}:初始猜测值,是一个数组。
    \item \texttt{args}:传递给目标函数和约束函数的额外参数。
    \item \texttt{method}:优化算法,可选值有 \texttt{'Nelder - Mead'}、\texttt{'Powell'}、\texttt{'CG'}、\texttt{'BFGS'}、\texttt{'Newton - CG'}、\texttt{'L - BFGS - B'}、\texttt{'TNC'}、\texttt{'COBYLA'}、\texttt{'SLSQP'} 等。
    \item \texttt{jac}:目标函数的雅可比矩阵(一阶导数)。
    \item \texttt{hess}:目标函数的海森矩阵(二阶导数)。
    \item \texttt{hessp}:海森矩阵与向量的乘积。
    \item \texttt{bounds}:变量的边界约束。
    \item \texttt{constraints}:约束条件。
    \item \texttt{tol}:收敛精度。
    \item \texttt{callback}:每次迭代后调用的回调函数。
    \item \texttt{options}:优化算法的选项。
\end{itemize}

\subsection{示例问题}
考虑一个简单的非线性规划问题:

最小化目标函数
\[
f(x_1, x_2) = (x_1 - 1)^2 + (x_2 - 2.5)^2
\]

约束条件:
\begin{align*}
x_1 - 2x_2 + 2 &\geq 0 \\
-x_1 - 2x_2 + 6 &\geq 0 \\
-x_1 + 2x_2 + 2 &\geq 0 \\
x_1 &\geq 0 \\
x_2 &\geq 0
\end{align*}

以下是使用 \texttt{scipy.optimize.minimize} 函数解决该问题的 Python 代码:
\begin{lstlisting}[style=pythonstyle]
import numpy as np
from scipy.optimize import minimize

# 定义目标函数
def objective(x):
    return (x[0] - 1) ** 2 + (x[1] - 2.5) ** 2

# 定义约束条件
constraints = [
    {'type': 'ineq', 'fun': lambda x: x[0] - 2 * x[1] + 2},
    {'type': 'ineq', 'fun': lambda x: -x[0] - 2 * x[1] + 6},
    {'type': 'ineq', 'fun': lambda x: -x[0] + 2 * x[1] + 2}
]

# 定义变量的边界约束
bounds = [(0, None), (0, None)]

# 初始猜测值
x0 = np.array([2, 0])

# 使用 SLSQP 方法进行优化
result = minimize(objective, x0, method='SLSQP', bounds=bounds, constraints=constraints)

# 输出结果
if result.success:
    print("最优解找到!")
    print(f"最优解: {result.x}")
    print(f"最优值: {result.fun}")
else:
    print("优化过程失败。")
\end{lstlisting}

通过上述代码,我们可以利用 \texttt{scipy.optimize.minimize} 函数有效地解决非线性规划问题。不同的问题可能需要选择不同的优化算法和设置合适的参数。

\end{document}